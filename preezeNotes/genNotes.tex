IID
==
To justify the i.i.d assumption, one must assume that states (or dyads) can have similar alliance generating processes based on similar values of their covariates, but states do not influence one another (e.g. the dyadic allianxce between the US and UK during the Cold War was entirely unrelated to other contemporaneous alliances). Generally, in the face of misspecification, the properties of MLE (asymptotic efficiency and unbiasedness) no longer hold (Greene, 2003).

The core of the problem is that regression models assume the data are independent and identically distributed (i.i.d) conditional on the covariates. The "independent" part of the i.i.d assumption implies that the observed relationships in the data constitute an exchangeable random sample. In other words, the labels on the observations, unless used to define independent variables, contain no relevant information. As a result, we should be able to shuffle the rows of a data matrix without affecting the inferences drawn from the data. Observations in the dataset can exchange places without affecting inference because each observation is assumed to be independent of every other observation. More specifically, the joint likelihood of u conditional on the outcome y and data X can only be established by assuming exchangeability and thus independence:

SAOM
==
At any given moment, with a given current network structure, the actors act independently without coordination. They also act one at a time. 

In other words, tie changes are not coordinated and depend on each other only sequentially, according to the changing configuration of the whole network. According to these four assumptions, the evolution process can be decomposed into its smallest possible components, which are called micro-steps.

This hypothesis allows us to consider the dependencies between network ties as the result of processes where one tie is formed as a reaction to the existence of other ties.

“This implies that tie changes are not coordinated, and depend on each other only sequentially, via the changing configuration of the whole network”. This implies strong dependence between what the actors do, but it is completely generated by the time order: the actors are dependent because they constitute each other's changing environment.

The model assumes a continuous Markov evolution of the network and decomposes the observed changes into the smallest possible components, i.e., modifications of one tie or one person’s practice at each time step between observations.

Furthermore, each component of a micro-step can be modeled by a specific process, so that the actor-oriented process can be decomposed into two stochastic sub-processes, which will be described in the next paragraphs. The first one is the change opportunity process which models the frequencies of the changes, in terms of the waiting time between a change to another, while the second one is the change determination process which models the precise tie made by the actor who has the opportunity to change.

At each micro-step one probabilistically selected actor has the opportunity to change. He/She can decide to change or not to change one of his outgoing ties, so that the utility function is maximized. Formally we can describe each micro-step as a pair of elements: the first determines the waiting time between one opportunity to change and the next one, while the second determines the precise change which is made. The set of the all micro-steps represents the so-called complete data, which are the result of the latent process underlying the evolution of the network. Thus, we cannot observe them, but only the final result. 

The set of the all micro-steps represents the so-called complete data, which are the result of the latent process underlying the evolution of the network. Thus, we cannot observe them, but only the final result.

Identification of the SAOM derives from the key assumption that when i takes a ministep (i.e., creates a new network tie), it does so independently of all other actors, conditional on the current structure of the network and exogenous covariates. That is, the model assumes that simultaneous influences are not actually "instantaneous". 

Rather, causality rests on temporal sequencing, wherein an actor first takes account of the structure of the network, and then changes its network tie accordingly. That change then becomes part of the network and thus influences the ministeps of subsequent actors. The total network change between observation moments is simply the accumulation of these ministeps. 

The sik(x(i Y j)) are called effects, and they are relevant functions of the digraphs which are supposed to play a key role in the network evolution. In other words they represent the leading forces of the underlying process that governs network changes from an ob- servation moment to another. As will be shown later, examples of effects are reciprocity and transitivity, whose outcomes have already been encounter in Paragraph 1.3.3. It is fundamental to specify that network effects are aspects of the network as perceived by the focal actor i.

The strength of each effect is represented by the corresponding parameter βk, which should be estimated on the basis of the longitudinal network data observed. βk can as- sume any real values and can be interpreted as follows. If βk is equal to 0, it means that the corresponding effect plays no role in the network dynamics. If it assumes positive values, then there is higher probability of moving into networks where the correspond- ing effect is higher. Vice versa if the parameter takes negative values there is higher probability of moving into networks where the corresponding effect is lower. Thus, for instance, if the parameter βk is related to the reciprocity effect and it takes a positive value, this means that the number of reciprocal dyad increases between two observation moments, so that there is evidence towards reciprocity.

Let us focus on an actor i, who can change one of his outgoing ties at a certain given moment, and analyze which are the possible choices of i. Given the current state of the network, i can decide not to change anything or to change one of his outgoing ties, for instance the tie xij directed to an actor j, into its opposite. Since we are considering sim- ple digraphs, and a tie can assume values 1 or 0, according to the fact that it is present or it is absent, changing a tie into its opposite means that the tie variable changes from 1 to 0 or from 0 to 1. In the first case the tie is terminated, while in the second the tie is created. Thus, if a relation between i and j exists in the current state of the network (xij = 1) and i decides to change it, the considered tie is deleted (xij = 0). Vice versa the tie is created.

This suggests that the set of admissible choice has cardinality equal to g: g − 1 changes and 1 non-change. Consequently, the set of possible states for the considered network given the current state contains g elements: g−1 network which are equals to the current state except for the value assumed by the changed tie and 1 equal to the current state. As previously said it turns out that each actor can choose between a discrete finite set of alternatives, which are mutually exclusive (since the selected actor can make only one change) and exhaustive, since he can decide among all the other actors having full knowledge of the entire network.

It is helpful to note that Eθ{gn(θ;zn)} does not depend on zn, so it is not random. 

The method of moments involves finding the moment estimate θˆ solving the moment equation

Finally,theestimation holds β constant at βˆ and performs additional simulations to estimate the covariance matrix, covβˆ, using a likelihood ratio / score function variant of the delta method, which Schweinberger and Snijders (2007) show to be unbiased, N-consistent, and much less computationally intensive than, e.g.,bootstrappingorresampling.

Since the estimates of βh generated by the simulation are approximately normally distributed (Snijders, van de Bunt, and Steglich 2010), null hypotheses can be tested with a simple t-statistic, th = βˆh , in the standard normal distribution (Snijders s.e.(βˆh ) 2005: 238).