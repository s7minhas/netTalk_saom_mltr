====================================================================================
====================================================================================
What are BITs
===
Next, I'm going to apply this framework to studying a particular problem in IPE that has received a notable amount of attention in recent years, specifically, I will be focusing on the evolution of the bilateral investment treaty network, BIT network. 

A BIT is an agreement establishing the terms and conditions for private investment by nationals and companies of one state in another state. So its an agreement signed between two countries that allows the companies in those countries investing in the other some set of property rights over their investment.

BITs typically ban discriminatory treatment against foreign investors, and what is special about them is that they even go so far as to provide guarantees of compensation for expropriated property or funds. 

And most importantly the adjudication for the violation of these agreements takes place not in the national courts of the violating country, but in international arbitration tribunals, such as the ICSID.

Legal scholars note that these agreements now provide the edifice for “a common lexicon of investment treaty law”. Transnational actors, such as multinational corporations (MNCs), have taken notice as they increasingly use BITs as a tool to assert their property rights in both developed and developing countries. 

And in recent years we have seen that MNCs have been able to use the dispute settlement mechanism in many of these agreements to challenge public policy regulations that adversely affect their investments. 
====================================================================================
====================================================================================

====================================================================================
====================================================================================
Why are they important
===
And this is exactly why they are important.

Now what makes BITs particularly cool or nefarious is that the provisions they provide for the protection of foreign investment were quite vague. and the implications of this have become much clearer in recent years. 

For example, one of the most prominent set of cases in recent years involves Philip Morris. And what PM was actually doing was picked up on in a talk show hosted by John Oliver on HBO. 

Now what he had picked up on was the fact that when the government of Uruguay passed tobacco regulations prohibiting cigarette makers from marketing more than one product under a single brand name and requiring that 80\% of cigarette cartons are covered with health warnings. Philip Morris almost immediately requested arbitration at the ICSID under a 1991 BIT between Uruguay and Switzerland. In their request for arbitration, Philip Morris alleged that these pieces of regulation significantly harm the sale of their products and constitute a failure to respect the company’s intellectual property rights. This case is currently ongoing. 

Togo, Namibia, Solomon Islands was sued by Philip Morris when it tried to add warning labels to cigarette cartons.

the threat of these types of lawsuits sgainst relatively poor countries raised concern at orgs like Bloomberg Philanthropies and Gates Foundation, who now have set up a multimillion dollar  Anti-Tobacco Trade Litigation Fund to help countries adjudicate these disputes.

Developed countries have dealt with legal challenges stemming from these agreements as well. Following Australia’s enactment of the Tobacco Plain Packaging Act in 2011, Philip Morris requested arbitration at ICSID under the Hong Kong-Australia BIT. The case is currently ongoing and Philip Morris is asking for compensatory damages in the “order of billions of Australian dollars” 

concerns about how far BITs can go in challening domestic policy objectives has raised fears even in Germany. Germany and Pakistan were actually the first pair of countries to sign a BIT in the 1960s, and recently the Economic Minister of Germany warned that many in his country fear that through ISDS provisions “states could be pressur[ed] and policy objectives circumvented by the threat of damages”.

% A company was able to sue a country over a public health measure.
====================================================================================
====================================================================================

====================================================================================
====================================================================================
Proliferation
===
The significant monetary costs and threats to domestic policy objectives pose an obvious question of the BIT regime, namely, what were the dynamics under which these agreements proliferated in the international system from 385 in 1989 to almost 3,000 by 2015?

This graphic here shows the development of the BIT network.

For the remainder of this talk, I'm going to focus on first briefly laying out the two major arguments for BIT formatin in the extant literature. And will then contrast those arguments with one alternative explanation. However, the major purpose of this empirical exercise is to retest the extant arguments in the literature using the network based approach that we discussed earlier. The existing literature in IPE has only evaluated this argument using a dyadic design in which they assume that the set of dyadic relationships in the international system are independent. 

Given my talk up until this point, the results in the extant literature have likely failed to adequately get at the data generating process because they have not used a network bsaed approach.
====================================================================================
====================================================================================

====================================================================================
====================================================================================
Credible Commitment
===
Now the first major argument in the literature revoles around the idea that countries signed BITs in order to make a credible commitment, so that they would receive FDI from investors in developed countries. 

Elkins et al. (2006) note that in recent decades BITs have become “the most important international legal mechanism for the encouragement and governance” of FDI into developing countries. Buthe & Milner echo this point and add that BITs boost FDI because they enshrine commitments to open markets and liberal economic policies from the developing countries that sign them. 

Given this fraiming scholars have argued that only “governments with little inherent crediblity are more likely to sign BITs”. This presumes that developing countries with legal systems that are already perceived to have impartial and fair property right regimes are less likely to sign BITs, as they need not provide any additional signals of their credibility. 

And the reason countries will only sign these agreements when they have something to prove is because scholars presume that countries have carefully considered the ex post costs of violating them. The implication being that only states willing and needing to make a credible commitment will participate. However, if states were seeking to make these strong credible commitments, then why are so many seeking to renegotiate the BITs that they have signed?

And at the least there's some anecdotal evidence to suggest otherwise....

To test the credible commitment hypothesis, I follow the extant literature in turning to the “Law and Order” measure from the International Country Risk Guide (ICRG) dataset. To determine ratings for their various measures, ICRG staff use political information and economic data to provide an assessment of the political and economic risks faced by countries on a variety of dimensions. 

The “Law and Order” measure is meant to provide an assessment of the strength and impartiality of a country’s legal system, as well as, the extent to which the law is observed. This variable ranges from 0 to 6, where higher ratings correspond to states with stronger legal systems whose decisions are followed. 

And to test the credible commitment argument, I use the country level scores to construct a dyadic covariate that is equal to the absolute value of the difference in ratings that an ij pair received on this measure in a given year. According to the credible commitment hypothesis, pairs of countries that have larger absolute differences on this score will be more likely to conclude a BIT with each other.
====================================================================================
====================================================================================

====================================================================================
====================================================================================
Competition for capital
==
The second primary argument for the proliferation of BITs is the competition for capital hypothesis suggested by Elkins et al. (2006) and refined by Plumper & Neumayer (2010). At its core the argument suggests that BIT formation is a result of capital importing countries (developing countries) competing with one another to attract FDI from capital exporters (developed countries). More specifically, BIT formation between a capital importer, i, and a capital exporter, j, is more likely when other capital importers, ∈/ i, sign BITs with j. Thus developing countries competing for capital feel pressure to sign BITs because not doing so could divert FDI flows from their countries.

This argument implies the development of a very specific structure in the BITs network. First, BITs between developing countries should be relatively rare. They are not, over a majority of BITs today are between developing countries. 

Second, the type of network that we would expect to see under this argument would have a hub and spoke structure. Where developed countries form the hubs and developing countries the spokes. This is not at all what we actually see develop. 

To test the competition for capital argument, I follow the procedure introduced by Neumayer & Plumper (2010) for constructing diffusion measures from dyadic data. 

The weighting matrix they use to construct each is a measure of the degree to which countries export a similar basket of goods. In this paper, I only show results using the weighted sum of BITs signed by other capital-importing target countries with the same capital-exporting source country. 

The argument presented under this framework is that a capital-importing country is more likely to sign a BIT with a capital exporter only if other competing capital importers have signed BITs with this very same capital exporter.

The argument presented under the competition for capital framework is that if a set of countries, ijk, are competing for FDI with a country, d, and k ratifies an agreement with d then the likelihood of i and j ratifying an agreement with d increases.
====================================================================================
====================================================================================

====================================================================================
====================================================================================
Political Affinity
==
Often ignored in studies of how and why international agreements, such as BITs, form is the political dimension. Though there may be a hope that certain economic transfers result from the agreement, the motivations for signing are often not just about investment or trade. 

In policy and press briefings, the idea that countries pursue the formation of economic agreements in order to fulfill political objectives is commonplace. Former Secretary of State, Hillary Clinton, referred to the political dimension motivating the formation of treaties governing economic relations as “economic statecraft” (Clinton, 2011). 

The Japanese prime minister, Shinzo Abe, in a speech in front of the United States Congress remarked that the Trans-Pacific Partnership Agreement (TPP) is ultimately about a deepening in existing political relationships, and that “its strategic value is awesome”.

The use of economic agreements in this way is a factor in why BITs propagated through the international system so rapidly. BITs are an ever easier method through which states can affirm and possibly strengthen their existing interstate political relationships. Easier because the wording of many of the provisions within BITs during their most popular phase was relatively consistent, as a result, very little negotiation was necessary (Montt, 2009).

They develop a dynamic ordinal spatial model to estimate state ideal points from along a single dimension, from which I calculate the difference in ideal points between countries by year.

% For example, officials from South Korea note that they did not foresee the implications BITs would have and that they had often simply replicated the texts of previous agreements that they had signed (Kim, 2011). Even one of the more controversial provisions, the “fair and equitable treatment” obligation, is “a standard that is repeated, more or less identically, in most of the. . . 2500 investment treaties in force at present”.
====================================================================================
====================================================================================

====================================================================================
====================================================================================
Endogenous explanations
==
Explicit network based effects. 

Indirect Ties
==
As the number of k third parties with whom i and j share agreements grows, information mechanisms reduce the risk of bilateral cooperation while externality mechanisms increase the relative benefits, generating third party incentives for direct ij cooperation. 

The more agreements a state enters, the more information it reveals about its preferences over institutional designs. So shared indirect ties between a pair of countries may convey information to each about what the other is willing to accept, and specifically, the various ties that a given country has are reflective of that country’s preferences with regards to, for instance, the role of ISDS in BITs. Thus I expect that two countries sharing a high number of indirect ties will be substantively more likely to sign a BIT.

% Though, in general, BITs almost universally enshrine the “same set of core substantive standards of treatment” for foreign investment (Montt, 2009), there are notable differences in these treaties by the inclusion of investor-state dispute settlement (ISDS) provisions (Yackee, 2008). Venezuela, for example, terminated its BIT with the Netherlands in early 2008, and since then Venezuela has only had BITs come into force with countries which agreed to not include provisions enabling investors to take arbitration claims to the ICSID. 

% A more careful approach to forming BITs is not unique to Venezuela, but a part of larger trend between both developed and developing countries (Haftel & Thompson, 2013; Manger & Peinhardt, 2013). India, for instance, is turning to only signing BITs with countries that agree to exclude provisions, which India sees as, unduly privileging foreign investors (Malik, 2008).  

Popularity
==
In particular, states may prefer to cooperate with partners that simply form more bilateral agreements in general, irrespective of whether they share partners in common. In terms of proposed causal mechanisms, high degree targets may convey trustworthiness and reliability more credibly than their low degree counterparts. 
====================================================================================
====================================================================================